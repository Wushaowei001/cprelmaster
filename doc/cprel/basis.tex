%\enableregime[utf-8]

\setupexternalfigures[directory={figures/}]
\useexternalfigure[follow][follow][width=.5\textwidth]

\starttext
\title{Preliminaries}

\startitemize
\item Let's define a finite set of integers  $\cal{U} = \{x: 0 \leq x
  \leq k \}$ for some large enough $k$.
\item A tuple $t$ is an ordered list of elements of $\cal{U}$ of the
  form $t = <t_{1}, t_{2}, \ldots, t_{i}, \ldots, t_{n}>$ where
  $t_{i}\in\cal{U}$. The arity of a tuple is the number of elements it
  contains: $\text{arity}(t)=n$.
\item A relation $R$ is a set of tuples of the same arity. There are
  several properties in a relation that are of our interest
  \startitemize
  \item The arity of the relation is the arity of the tuples it
    contains: $\text{arity}(R)=\text{arity}(t)$, where $t$ is a tuple
    of the relation.
  \item The cardinality of $R$, represented as $\text{card}(R)$, is the
    number of tuples it contains.
  \item Another way of looking at $R$ is by columns. The number of
    columns in $R$ is the same as its arity.
  \stopitemize
\stopitemize

\section{Follow}
Te parameters of the follow constraint are three relation decision
variables $A\subseteq\cal{U}_{n}$, $B\subseteq\cal{U}_{m}$ and
$C\subseteq\cal{U}_{k}$; and an integer representing the length of the
match: $f \in \{x : 1 \leq x \leq \text{min}(n,m)\}$. The semantincs
of the constraint is defined by:
\startformula
A
\stopformula

\externalfigure[follow]

\stoptext